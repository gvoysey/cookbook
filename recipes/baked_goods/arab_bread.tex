\begin{recipe}{Lebanese Flatbread}{Vincent Zeng}{1 hour}
	\ing{1 1/2}{lb}{flour}
	\ing{1}{tsp}{salt}
	\ing{1 7/8}{cup}{warm water with yeast}

	In a large bowl, mix dry ingredients. Make a well and pour in yeasty water, gently mixing a little with a fork until most of the water is mixed in. Roll up your sleeves. Knead firmly until the dough feels homogenous and tight; this shouldn't take more than 15 minutes. Cover with a damp cloth and let rise in a warm, quiet place (I recommend in the oven with only the oven light on). Rise times vary depending on yeast and environmental factors; instant yeast can take as little as 4 hours, while live yeast can take up to 12. Rise times beyond 18 hours are not recommended, as the yeast will eventually consume so much flour that the dough will not hold together.

	Separate the dough into twelve roughly equal chunks. Roll on a well-floured surface; the thinner the dough, the more it will puff up and get crunchy. It should not be so thin that it doesn't hold together. Lay the rolled dough onto a foil or parchment paper covered wire rack and broil on high for about 5-10 minutes, until the tops rise and become golden brown. Remove from the baking surface immediately and flip them upside-down to cool.
\end{recipe}
\textit{Serve fresh, or wrap in a cloth and store in a breadbox for a few days.}
